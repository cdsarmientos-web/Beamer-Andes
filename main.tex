
\documentclass{beamer}
%Tomado de https://bioinformatiquillo.wordpress.com/2008/10/08/temas-para-presentaciones-de-la-clase-latex-beamer/
\usepackage{color}
\usepackage[english]{babel}
%\usepackage[pdftex]{graphicx}
%\usepackage[pdftex]{hyperref}
\usepackage{amsmath,amsthm,amscd}
\usepackage{amsmath,amssymb,amsthm,graphicx, mathrsfs}
\usepackage[all]{xy}
\usepackage[utf8]{inputenc}
\usepackage{lscape}
\usepackage{fancyhdr}
\usepackage{amsfonts}
%\usepackage{fontspec}
\usepackage{pb-diagram}
\usepackage{tikz-cd}
\usepackage{multicol}
\usepackage{url}
\usetheme{Boadilla}
\usefonttheme{serif}%para que se ponga la fuende por defecto d elos pdf de Latex
%\setmainfont{DarwinFont}

%1) Antibes
%2) bars
%3) Bergen
%4) Berkeley
%5) Berlin
%6) Boadilla
%7) boxes
%8 ) classic
%9) Copenhagen
%10) Darmstadt
%11) default
%12) Dresden
%13) Frankfurt
%14) Goettingen
%15) Hannover
%16) Ilmenau
%17) JuanLesPins
%18) lined
%19) Luebeck
%20) Madrid
%21) Malmoe
%22) Marburg
%23) Montpellier
%24) PaloAlto
%25) Pittsburgh
%26) Rochester
%27) shadow
%28) sidebar
%29) Singapore
%30) split
%31) Szeged
%32) Tree
%33) Warsaw

\usecolortheme{default}

%1) albatross (azul marino)
%2) beetle (azul y gris)
%3) crane (naranja y blanco)
%4) default (por defecto)
%5) dolphin (azul marino y blanco)
%6) dove (gris y blanco)
%7) fly (gris)
%8 ) lily
%9) orchid
%10) rose
%11) seagull (gris y blanco)
%12) seahorse
%13) sidebartab
%14) structure
%15) whale (azul marino y blanco)


%\newtheorem{theorem}{Theorem}[section]

\newtheorem{acknowledgement}[theorem]{Acknowledgement}
\newtheorem{proposition}[theorem]{Proposition}
%\newtheorem{lemma}[theorem]{Lemma}
\newtheorem{algorithm}[theorem]{Algorithm}
\newtheorem{axiom}[theorem]{Axiom}
\newtheorem{case}[theorem]{Case}
\newtheorem{claim}[theorem]{Claim}
\newtheorem{conclusion}[theorem]{Conclusion}
\newtheorem{condition}[theorem]{Condition}
\newtheorem{conjecture}[theorem]{Conjecture}
%\newtheorem{corollary}[theorem]{Corollary}
\newtheorem{criterion}[theorem]{Criterion}


\theoremstyle{definition}
%\newtheorem{example}[theorem]{Example}
%\newtheorem{definition}[theorem]{Definition}
%\newtheorem{examples}[theorem]{Examples}
\newtheorem{exercise}[theorem]{Exercise}
\newtheorem{notation}[theorem]{Notation}
%\newtheorem{problem}[theorem]{Problem}
\newtheorem{remarks}[theorem]{Remarks}
\newtheorem{remark}[theorem]{Remark}
%\newtheorem{solution}[theorem]{Solution}
\newtheorem{summary}[theorem]{Summary}




\usepackage{graphicx} % Required for inserting images
\usepackage{color}
\usepackage[all]{xy}
\usepackage{multicol}

\newtheorem{definicion}{Definición}
\newtheorem{ejemplo}{Ejemplo}
\newtheorem{teorema}{Teorema}


\title[Pushforward structures]{{\bf{\large Pushforward structure to relate geometric cycles}}}
\author[C.D.S.S]{Cristian D. Sarmiento Santiago\\\vspace{0.1cm} {\footnotesize Universidad Nacional de Colombia}\\\vspace{0.4cm}
A joint work with M. Velasquez and P. Carrillo.}
\date[]{Jornadas de Análisis Global, Geometría y Física\\ November 27, 2025}

\begin{document}

\maketitle


\begin{frame}{Contents}
\tableofcontents    
\end{frame}
%%%%%%%%%%%%%%%%%%%%%%%%%%%%%%%%%%%%%%%%%%%%%%%%%%



\section{Motivation and main problem}
\begin{frame}{Motivation}
    \begin{figure}
        \centering
        \includegraphics[scale=0.5]{Portada.png}
        %\caption{Caption}
        %\label{fig:enter-label}
    \end{figure}
\end{frame}

\begin{frame}{Problem of interest}
    \begin{figure}
        \centering
        \includegraphics[scale=0.8]{Auxiliar.png}
        \caption{Auxiliary homology theory}
        %\label{fig:enter-label}
    \end{figure}
\end{frame}


\section{Main definitions}


\begin{frame}{Basic definitions}
A continuous map $f:X\to Y$ is called {\bf proper} if for every compact $K\subseteq Y$, we have that $f^{-1}(K)$ is a compact of $X$. \pause

\vspace{0.3cm}

The {\bf action} $\rho: G\times X\to X:(g,x)\mapsto g\cdot x$ is called {\bf proper} ($X$ locally compact and Hausdorff) if the map is a proper continuous function.$$G\times X\to X\times X:(g,x)\mapsto (g\cdot x,x)$$ \pause

\vspace{0.3cm}

{\bf Classifying space}: $X$ a proper $G$-space, then there exists a $G$-map $$\phi_{X}:X\to \underline{E}G,$$ and any two $G$-maps from $X$ to $\underline{E}G$ are $G$-homotopic.

\end{frame}

\begin{frame}{Groupoid and Groupoid actions}

    \begin{itemize}
    \item A topological groupoid is a groupoid $\mathcal{G}$ such that $Z$ and $\mathcal{G}$ are Hausdorff and the structure maps source and target $s,t:\mathcal{G}\to Z$ are continuous and open.
        \item Let $\mathcal{G}$ be a topological groupoid. A topological space $X$ is called a {\bf $\mathcal{G}$-space} if there is a continuous map $a:X\to Z$ called the {\bf anchor} map and a homeomorphism
        \begin{align*}
             \mathcal{G}_{s}\times_{a} X\to \mathcal{G}_{t}\times_{a} X: (g,x)\mapsto (g,g\cdot x),
        \end{align*}
        which determines the {\bf action}, and holds associativity and unitality conditions.
    \end{itemize}
\end{frame}


\begin{frame}{Groupoid examples}
\begin{example}
For a group $\Gamma$, consider the groupoid $\mathcal{G}:=\Gamma^{*}$, where $\mathcal{G}^{0}:=\{*\}$ and $\mathcal{G}^{1}:=\Gamma$.
\end{example}\pause

\begin{example}
    The transformation groupoid $\mathcal{G}\ltimes X$ (see \cite{EM10}, p. 4, Definition 2.2) is defined by $X$ as its object space, morphisms the elements of $\mathcal{G}_{s}\times_{a} X$; the range and source maps are defined by $r(g,x)= g\cdot x$ and $s(g,x)= x$  respectively ; and its composition  is given by $(g,x)\cdot (h,y) = (gh,y)$.
\end{example}

\end{frame}


\begin{frame}{$(\Gamma^{*}\ltimes X)$-bundles}
When $\mathcal{G}$ is the transformation groupoid (or action groupoid) of a group, we recover the notion of a $\Gamma$-vector bundle. For $\Gamma$ be a discrete group, and $X$ be a $\Gamma$-space, there is a bijection: 
\begin{align*}
\{\Gamma{\rm -vector\ bundles\ over\ }X\}\leftrightarrow \{(\Gamma^{*}\ltimes X){\rm -vector\ bundles\ over\ }X\} 
\end{align*}\pause
$(\leftarrow):$ For a $(\Gamma^{*}\ltimes X)$-vector bundle $\rho:E\to X$, we note the $\Gamma$-vector bundle structure with action on $E$ given by
\begin{align*}
\gamma e:= (\gamma, \rho(e))e, 
\end{align*}\pause
\noindent $(\rightarrow):$ If $\rho:E\to X$ is a $\Gamma$-vector bundle we note that it has a $(\Gamma^{*}\ltimes X)$-vector bundle structure by the anchor map $a:=\rho:E\to X$, and the natural action
\begin{align*}
(\gamma, x)e:= \gamma e\ \ \ when\ s((\gamma,x))=a(e),\ i.e.,\ e\in E_{x}=\rho^{-1}(x)
\end{align*}

\end{frame}

\begin{frame}{Emerson-Meyer}
Let $\mathcal{G}$ be a proper groupoid, let $X$ and $Y$ be smooth $\mathcal{G}$-manifolds, and let $f:X\to Y$ be a smooth $\mathcal{G}$-equivariant map (+).\pause
    Then, there are
    \begin{itemize}
        \item a smooth $\mathcal{G}$-vector bundle $V$ over $X$,
        \item a smooth $\mathcal{G}$-vector bundle $E$ over $Z$,
        \item a smoth $\mathcal{G}$-equivariant, open embedding $\eta_{f}:V\to E^{Y}$,
    \end{itemize}
    such that $f=\rho_{E^{Y}}\circ \eta_{f}\circ \xi_{V}$, where $\xi_{V}:X\to V$ is the $zero$-section of the vector bundle $\rho_{V}:V\to X$. \pause
    \begin{align*}
        &\xymatrix{V\ar[r]^{\eta_{f}} & E^{Y}\ar[d]^{\rho_{E^{Y}}}& E\ar[d]\\
	{X}\ar[r]_{f}\ar[u]^{\xi_{V}} & Y\ar[r]_{a}& \underline{E}\Gamma}
    \end{align*}
\end{frame}




\begin{frame}{Main definitions}

{\bf Cohomology Theory:} $\mathscr{H}_{\Gamma}^{*}:(Proper\ Pairs)\to (R-mod)$ with connection maps $\partial^{n}_{\Gamma}:\mathscr{H}_{\Gamma}^{n}(A)\to\mathscr{H}_{\Gamma}^{n+1}(X,A)$ with axioms
\begin{itemize}
   \item $\Gamma$-homotopy invariance: If $f$ and $f'$ are homotopics then $f^{*}=f'^{*}$
    \item Long exact sequence: For $i:A\to X$ and $j:X\to(X,A)$, 
    \begin{align*}
    \mathscr{H}_{\Gamma}^{n}(X,A)\xrightarrow{j^{n}}\mathscr{H}_{\Gamma}^{n}(X)
    \xrightarrow{i^{n}}\mathscr{H}_{\Gamma}^{n}(A)\xrightarrow{\partial^{n}_{\Gamma}}\mathscr{H}_{\Gamma}^{n+1}(X,A)
    \end{align*}
   
   \item Excision: For $(X,A)$ a $\Gamma$-proper-pair and $f:A\to B$, the canonical map $F:(X,A)\to( X\cup_f B,B)$ induces an isomorphism
    $$F^{*}:\mathscr{H}_{\Gamma}^{n}(X,A)\rightarrow\mathscr{H}_{\Gamma}^{n}(X\cup_f B,B).$$
    \item Disjoint union: the following map is a group isomorphism:
    $$\prod_{i\in I}j_{i}^{*}:\mathscr{H}_{\Gamma}^{n}(\coprod_{i\in I}X_i) \rightarrow    \prod_{i\in I}\mathscr{H}^{n}_{\Gamma}(X_i)$$
    
\end{itemize}
    
\end{frame}





\section{Principal tool}



\begin{frame}
\begin{definition}


    For a $d$-dimensional real $\Gamma$-vector bundle  $E\to X$ over a proper $\Gamma$-space $X$, a $\mathscr{H}_{\Gamma}$-orientation for $E$ is a class $\tau\in \mathscr{H}_{\Gamma}^{d}(E)$ such that for each proper $\Gamma$-map $f:Y\to X$, multiplication with $f^{*}\tau \in \mathscr{H}_{\Gamma}^{d}(f^{*}E)$ induces an isomorphism (refered as Thom-isomorphism)
    \begin{align*}
        -\otimes \tau:\mathscr{H}_{\Gamma}^{*}(Y)\to \mathscr{H}_{\Gamma}^{*+d}(f^{*}E).
    \end{align*}
    We say that a smooth manifold with boundary $M$ is a $\mathscr{H}_{\Gamma}$-orientable if $TM$ is $\mathscr{H}_{\Gamma}$-orientable. When is chosen a fixed $\mathscr{H}_{\Gamma}$-orientation for a real $\Gamma$-vector bundle (or particularly for a manifold), we say that it is a $\mathscr{H}_{\Gamma}$-oriented bundle (or manifold).
\end{definition}
\end{frame}



\begin{frame}{Pushforward structure}
    For $f:(M,\partial M)\to (N,\partial N)$, 
    \begin{align*}
        f!:\mathscr{H}_{\Gamma}^{*}(M)\to \mathscr{H}_{\Gamma}^{*+dim(N)-dim(M)}(N)
    \end{align*}
    With axioms:\pause
    \begin{itemize}
        \item $(f\circ g)!=f!\circ g!$.\pause
        \item If $f$ and $f'$ are homotopic then $f!=f'!$\pause
        \item For $\rho:E\to M$ is a $\Gamma$-vector bundle, if $s:M\to E$ is the zero section, then the following map is a Thom isomorphism:
        \begin{align*}
            s!:\mathscr{H}_{\Gamma}^{*}(M)\to \mathscr{H}_{\Gamma}^{*+dim(E)-dim(M)}(E)
        \end{align*}\pause
        \item For a collar around $\partial N$ in $N$, $l!=\partial^{*}$ and $i^{0}!=j^{*}$
        \begin{align*} 
    \cdots\to\mathscr{H}_{\Gamma}^{*-1}(\partial N)\xrightarrow{l_!}\mathscr{H}_{\Gamma}^*(N^0)\xrightarrow{i^0_!}\mathscr{H}_{\Gamma}^*(N)\to\cdots
    \end{align*}
    \end{itemize}
    
\end{frame}


\begin{frame}{Equivalence relation}

\begin{itemize}
    \item $n$-Cycle over $(X,A)$: $\pi:(M,\partial M)\to (X,A)$ and $x_{M}\in \mathscr{H}_{\Gamma}^{n}(M)$.\pause
    \item Related cycles: $(M,\partial M,\pi_{M}, x_{M})\sim (N,\partial N,\pi_{N}, x_{N})$ if the following diagram is commutative and $f_{M}!(x_{M})=f_{N}!(x_{N})$:
    \begin{align*}\xymatrix{W\ar[dr]|{\pi_{W}} & N\ar[l]_{f_{N}} \ar[d]^{\pi_{N}} \\
    M\ar[u]^{f_{M}}\ar[r]_{\pi_{M}}& X }\end{align*}\pause
    \item Define $\mathscr{H}_{*}^{pf}(X,A;\Gamma)$ the set of equivalence relations of tuples $(M,\partial M,\pi_{M}, x_{M})$ where $x_{M}\in \mathscr{H}_{\Gamma}^{*+dim(M)}(M)$.
\end{itemize}
    
\end{frame}

\begin{frame}{Important Lemmas}
\begin{lemma}
    If $(M,\partial M, \pi_{M},x_{M})\sim (N,\partial N, \pi_{N},x_{N})$ by the maps $f_{M}:M\to W$ and $f_{N}:N\to W$, then we get that $(M,\partial M, \pi_{M},x_{M})\sim (N,\partial N, \pi_{N},x_{N})$ by maps $\phi_{M}:M\to E^{W}$ and $\phi_{N}:N\to E^{W}$ with $\phi_{N}$ being an embedding and $E^{W}\to W$ is a vector bundle. 
\end{lemma}\pause

\begin{lemma}
For any $\Gamma$-proper embeddings $i_{1}:N\to W_{1}$ and $i_{2}:N\to W_{2}$, there exist $\Gamma$-vector bundles $D_{1}\to W_{1}$ and $D_{2}\to W_{2}$ such that $$\nu_{\xi_{D_{1}}\circ i_{1}}\cong \nu_{\xi_{D_{2}}\circ i_{2}}.$$
\end{lemma}
\end{frame}



\begin{frame}{Equivalence relation (sketch of proof) I}
    \begin{figure}
        \centering
        \includegraphics[width=0.5\linewidth]{Relacion 1.png}
        \caption{Initial relations.}
        %\label{fig:enter-label}
    \end{figure}\pause
    \begin{align*}
    W:=(D_{1}\oplus \mathbb{R})_{\nu_{1}}\#_{\nu_{2}} (D_{2}\oplus\mathbb{R}),
\end{align*}
\end{frame}

\begin{frame}{Equivalence relation (sketch of proof) II}
\begin{figure}
    \centering
    \includegraphics[scale=0.5]{Relacion 2.png}
    \caption{$\Gamma$-vector bundles}
    %\label{fig:enter-label}
\end{figure}\pause
    $$D_{1}:=a_{1}^{*}E_{2}\oplus C_{1}\to W_{1}$$.
\end{frame}


\begin{frame}{Equivalence relation (sketch of proof) III}
\begin{figure}
    \centering
    \includegraphics[width=0.5\linewidth]{Relación 3.png}
    \caption{The desired connection space.}
    %\label{fig:enter-label}
\end{figure}
\end{frame}

\begin{frame}{Examples}
$[M,\partial M,\pi_{M}, x_{M}]\in\mathscr{H}_{*}^{pf}(X,A;\Gamma)$ with $x_{M}\in \mathscr{H}_{\Gamma}^{*+dim(M)}(M)$ $$\pi_{M}:M\to X$$
\begin{itemize}
    \item $\mathscr{H}_{*}^{pf}(\{a\};\Gamma)=\{[M,x_{M}]:\partial M=\emptyset \}$ ($|\Gamma|<\infty$). \pause
    \item $\mathscr{H}_{0}^{pf}(X,A;\Gamma)\in [S^{n},\emptyset, \pi_{S^{n}},\alpha]$, where $\pi_{S^{n}}:S^{n}\to X$ is an $n$-loop and $\alpha\in \mathscr{H}_{\Gamma}^{n}(S^{n})$. \pause
    \item For $k>0$, in $\mathscr{H}_{k}^{pf}(X,A;\Gamma)$, $[S^{n},\emptyset, \pi_{S^{n}}, \alpha]=0$, by $\mathscr{H}_{\Gamma}^{n+k}(S^{n})=0$ ($\Gamma=\{e\}$). 
\end{itemize}

\end{frame}

\begin{frame}{Swinging}
\begin{remark}\label{relation 1}
    If there is a proper smooth, $\Gamma$-equivariant, $\mathscr{H}_{\Gamma}$-orientation preserving map $f:(M, \partial M)\to (N,\partial N)$ such that $\pi_{N}\circ f=\pi_{M}$ and $f!(x_{M})=x_{N}$, then 
    $$[M,\partial M,\pi_{M},x_{M}]=[N,\partial N,\pi_{N},x_{N}],$$ i.e.,$$[M,\partial M,\pi_{N}\circ f,x_{M}]=[N,\partial N,\pi_{N},f!(x_{M})].$$

\end{remark}

\end{frame}

\section{Homology theory}

\begin{frame}{Functoriality and homotopic invariance}
\begin{multicols}{2}
    \begin{align*}
        &\text{For a map $g:(X,A)\to (Y,B)$},\\
        g_{*}&:\mathscr{H}_{*}^{pf}(X,A;\Gamma)\to \mathscr{H}_{*}^{pf}(Y,B;\Gamma):\\&[M,\partial M,\pi,x]\mapsto [M,\partial M,g\circ \pi,x]
    \end{align*}
\begin{align*}
    \xymatrix{&\\
              M\ar[d]_{\pi}&\\
              X\ar[r]_{g}&Y
    }
\end{align*}
\end{multicols}
\begin{enumerate}
    \item $(f\circ g)_{*}=f_{*}\circ g_{*}$
    \item $\Gamma$-homotopy invariance: if $g_{0}, g_{1}:(X,A)\to (Y,B)$ are proper $\Gamma$-homotopic maps of $\Gamma$-proper pairs, then
    \begin{align*}
        g_{0*}=g_{1*}: \mathscr{H}_{n}^{pf}(X,A;\Gamma)\to \mathscr{H}_{n}^{pf}(Y,B;\Gamma)
    \end{align*}
\end{enumerate}


    
\end{frame}

\begin{frame}{Long exact sequence}
      
\begin{align*}
    \xymatrix{
	\text{For the sequence of $\Gamma$-proper pairs}&(A,\emptyset) \ar[r]^{j}& {(X,\emptyset)} \ar[r]^{i}& {(X,A)}
	}
\end{align*}

The following sequence is exact 
\begin{align*}
    \xymatrix{\mathscr{H}^{pf}_{*}(A;\Gamma) \ar[r]^{j_{*}}& {\mathscr{H}^{pf}_{*}(X;\Gamma)} \ar[r]^{i_{*}}& {\mathscr{H}^{pf}_{*}(X,A;\Gamma)} \ar[r]^{\partial_{*}}& {\mathscr{H}^{pf}_{*-1}(A;\Gamma)}}
\end{align*}\pause
where
\begin{itemize}
    \item $j_{*}([M,\emptyset,\pi_{M},x_{M}])= [M,\emptyset,j\circ \pi_{M},x_{M}]$.
    \item $i_{*}([M,\emptyset,\pi_{M},x_{M}])=[M,\emptyset,\pi_{M},x_{M}]$.
    \item $\partial([M,\partial M,\pi_{M},x_{M}])=[\partial M,\emptyset,\pi_{M}|_{\partial M}, i_{\partial M}^{*}(x_{M})]$, with $i_{\partial M}:\partial M \hookrightarrow M$.
\end{itemize}

\end{frame}

\begin{frame}{{\bf Exactness in $\mathscr{H}_{*}^{pf}(X;\Gamma)
    $}}
     $Ker(i_{*})\subseteq Im(j_{*})$: Let $[M,\emptyset, \pi_{M}, x_{M}]=[N,\partial N, \pi_{N}, 0]\in \mathscr{H}^{pf}_{*}(X,A;\Gamma)$ with $f_{M}!(x_{M})=0$ with $f_{M}:M\to W$. $f_{M}!(x_{M})\in Ker(i_{W}^{0}!)$
     \begin{align*}
         \xymatrix{{\mathscr{H}_{\Gamma}^{*-1}(\partial W)}\ar[r]_{\partial} & {\mathscr{H}_{\Gamma}^{*}(W^{0})} \ar[r]_{i_{W^{0}}!} & {\mathscr{H}_{\Gamma}^{*}(W)}}
     \end{align*}\pause
     There exist $y\in \mathscr{H}_{\Gamma}^{*-1}(\partial W)$ such that $\partial(y)=f_{M}!(x_{M})$, and we get
     \begin{align*}
    j_{*}([\partial W, \emptyset,\pi_{W}|_{\partial W}, y])&=[\partial W, \emptyset,j\circ \pi_{W}|_{\partial W}, y]\\&=[\partial W, \emptyset,\pi_{W}\circ l, y]\\
    &=[W^{0}, \emptyset,\pi_{W}|_{W^{0}}, l!(y)]\\
    &=[W^{0}, \emptyset,\pi_{W}|_{W^{0}}, \partial(y)]\\
    &=[W^{0}, \emptyset,\pi_{W}|_{W^{0}}, f_{M}!(x_{M})]\\
    &=[M, \emptyset,\pi_{W}\circ f_{M},x_{M}]\\
    &=[M, \emptyset,\pi_{M},x_{M}].
\end{align*}
\end{frame}

\begin{frame}{{\bf Exactness in $\mathscr{H}_{*}^{pf}(X,A;\Gamma)
    $}}
$Ker(\partial)\subseteq Im(i_{*})$: Let $[M,\partial M,\pi_{M},x_{M}]\in Ker(\partial)\subseteq \mathscr{H}_{*}^{pf}(X,A;\Gamma)$, then, $[\partial M , \emptyset, \pi_{M}|_{\partial M}, r_{\partial}(x_{M})]=[W,\emptyset, \pi_{W},0]$    \pause
  \begin{align*}
 \xymatrix{M&M^{0}\ar[l]_{i_{M^{0}}}&{\partial M}\ar[r]_{f} \ar[l]_{l}& W}
	\end{align*}
	consider the embeddings $i_{M^{0}}\circ l:\partial M \to M$ and $f:\partial M \to W$, to get that there exist $\Gamma$-vector bundles $\rho_{D_{M}}:D_{M}\to M$ and $\rho_{D_{W}}:D_{W}\to W$ such that $\nu_{M}:=\nu_{\xi_{D_{M}}\circ i_{M^{0}}\circ l}\cong \nu_{\xi_{D_{W}}\circ f}=:\nu_{W}$, and 
\begin{align*}
    T:=(D_{M}\oplus \mathbb{R})_{\nu_{M}\oplus \mathbb{R}}\#_{\nu_{W}\oplus \mathbb{R}} (D_{W}\oplus\mathbb{R}),
\end{align*} \pause
$$[M,\partial M, \pi_{M}, x_{M} ]=[T,\partial T, \pi_{T},\iota!(x_{M})]=i_{*}([T^{0},\emptyset, \pi_{T}|_{T^{0}},y])$$ 
\end{frame}

\begin{frame}{{\bf Exactness in $\mathscr{H}_{*}^{pf}(A;\Gamma)
    $}}
$[M,\emptyset, \pi_{M}, x_{M}]\in Ker(j_{*-1})\subseteq Im(\partial)$
$$[M,\emptyset, j\circ\pi_{M},x_{M}]=[W,\emptyset,\pi_{W}, 0_{W}]\in\mathscr{H}_{*}^{pf}(X,\emptyset;\Gamma)$$\pause
\begin{align*}
    &\ \ \ \ \ \ \ \ \ \ \ \ \ \  \ \xymatrix{V\ar[r]^{\eta_{f}} & E\ar[d]^{\rho_{E}} \\
	{M}\ar[r]_{f}\ar[u]^{\xi_{V}} & W}\\
    T:&= DNC(E,M)|_{[0,1)}=V\times \{0\}\sqcup E\times(0,1)
\end{align*}\pause
    $$\partial([T,\partial T,\pi_{T}, y])=[M,\emptyset,\pi_{M},x_{M}]$$
\end{frame}

\begin{frame}{Excision and future work}
For a map $\phi:(X,A)\to (Y,B)$ such that $Y\cong X\cup_{A}B$, we get that $\mathscr{H}^{pf}_{*}(X,A)\cong \mathscr{H}^{pf}_{*}(X\cup_{A}B, B)$. \pause
\begin{align*}
    [M,\partial M,\pi_{M},x_{M}]\in\mathscr{H}^{pf}_{*}(X,A)\mapsto {[M,\partial M,\phi\circ\pi_{M},x_{M}]\in \mathscr{H}^{pf}_{*}(Y,B)}
\end{align*}
with inverse
\begin{align*}
    h([M,\partial M,\pi_{M},x_{M}])=[M_{U}=\pi_{M}^{-1}(U),\partial M_{U},r\circ\pi_{M}|_{M_{U}},i_{M_{U}}^{*}x_{M}].
\end{align*}\pause
Consider $\widetilde{W}=M\times(0,1]\cup M_{U}\times \{0\}$, and with this, define
    \begin{align*}
        W=\widetilde{W}\cup_{\partial M\times(0,1]\cup \partial M_{U}\times \{0\}}\widetilde{W}.
    \end{align*}   \pause
    \begin{align*}
        [M_{U},\partial M_{U},\phi\circ r\circ\pi_{M}|_{M_{U}},i_{M_{U}}^{*}x_{M}]
        &=[W^{0},\emptyset,\pi_{W^{0}}, i_{1}!(x_{M})]\\
        &=[M,\partial M,\pi_{M},x_{M}],
    \end{align*}  
\end{frame}




\begin{frame}{Bibliography}


    
\thebibliography{10}
\bibitem{EM10} Heath Emerson and Ralf Meyer. Equivariant embedding theorems and topological index maps. Advances in Mathematics,
225(5):2840–2882, 2010.

\bibitem{Lück05} Wolfgang Lück. Equivariant cohomological chern characters. International Journal of Algebra and Computation, 15(05n06):1025–1052, 2005.

\bibitem{DS19} Claire Debord and Georges Skandalis. Lie groupoids, pseudodifferential calculus, and index theory. Advances in Noncommutative
Geometry: On the Occasion of Alain Connes’ 70th Birthday, pages
245–289, 2019.

\bibitem[BCH94]{BCH94} Paul Baum, Alain Connes, and Nigel Higson. Classifying space for proper actions and K-theory of group $C^{*}$-algebras. Contemporary Mathematics, 167:241-241, 1994

\end{frame}

\begin{frame}{Bibliography}
\thebibliography{10}

\bibitem{BHS07} Paul Baum, Nigel Higson, and Thomas Schick. On the equivalence of geometric and
analytic K-homology. Pure Appl. Math. Q., 3(1):1–24, 2007.

\bibitem{BHS10} Paul Baum, Nigel Higson, and Thomas Schick. A geometric description of equivariant K-homology for proper actions. Quanta of maths 11 (2010): 1-22.


\bibitem{CWW22} Paulo Carrillo Rouse, Bai-Ling Wang, and Hang Wang. Topological K-theory for
discrete groups and index theory. arXiv, page 2012.12359, 2022

\bibitem{CW14} Paulo Carrillo Rouse and Bai-Ling Wang. Geometric Baum-Connes assembly map
for twisted differentiable stacks. arXiv preprint arXiv:1402.3456, 2014

\end{frame}


\begin{frame}

\begin{center}
   % 
   \begin{figure}
       \centering
       \includegraphics[scale=0.4]{Andes.jpg}
       \label{fig:enter-label}
   \end{figure}
\end{center}
    \begin{center}
        {{\huge Thanks a lot :)} }
    \end{center}
    
\end{frame}




\end{document}
